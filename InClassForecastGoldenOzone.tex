% Options for packages loaded elsewhere
\PassOptionsToPackage{unicode}{hyperref}
\PassOptionsToPackage{hyphens}{url}
%
\documentclass[
]{article}
\usepackage{amsmath,amssymb}
\usepackage{lmodern}
\usepackage{iftex}
\ifPDFTeX
  \usepackage[T1]{fontenc}
  \usepackage[utf8]{inputenc}
  \usepackage{textcomp} % provide euro and other symbols
\else % if luatex or xetex
  \usepackage{unicode-math}
  \defaultfontfeatures{Scale=MatchLowercase}
  \defaultfontfeatures[\rmfamily]{Ligatures=TeX,Scale=1}
\fi
% Use upquote if available, for straight quotes in verbatim environments
\IfFileExists{upquote.sty}{\usepackage{upquote}}{}
\IfFileExists{microtype.sty}{% use microtype if available
  \usepackage[]{microtype}
  \UseMicrotypeSet[protrusion]{basicmath} % disable protrusion for tt fonts
}{}
\makeatletter
\@ifundefined{KOMAClassName}{% if non-KOMA class
  \IfFileExists{parskip.sty}{%
    \usepackage{parskip}
  }{% else
    \setlength{\parindent}{0pt}
    \setlength{\parskip}{6pt plus 2pt minus 1pt}}
}{% if KOMA class
  \KOMAoptions{parskip=half}}
\makeatother
\usepackage{xcolor}
\usepackage[margin=1in]{geometry}
\usepackage{color}
\usepackage{fancyvrb}
\newcommand{\VerbBar}{|}
\newcommand{\VERB}{\Verb[commandchars=\\\{\}]}
\DefineVerbatimEnvironment{Highlighting}{Verbatim}{commandchars=\\\{\}}
% Add ',fontsize=\small' for more characters per line
\usepackage{framed}
\definecolor{shadecolor}{RGB}{248,248,248}
\newenvironment{Shaded}{\begin{snugshade}}{\end{snugshade}}
\newcommand{\AlertTok}[1]{\textcolor[rgb]{0.94,0.16,0.16}{#1}}
\newcommand{\AnnotationTok}[1]{\textcolor[rgb]{0.56,0.35,0.01}{\textbf{\textit{#1}}}}
\newcommand{\AttributeTok}[1]{\textcolor[rgb]{0.77,0.63,0.00}{#1}}
\newcommand{\BaseNTok}[1]{\textcolor[rgb]{0.00,0.00,0.81}{#1}}
\newcommand{\BuiltInTok}[1]{#1}
\newcommand{\CharTok}[1]{\textcolor[rgb]{0.31,0.60,0.02}{#1}}
\newcommand{\CommentTok}[1]{\textcolor[rgb]{0.56,0.35,0.01}{\textit{#1}}}
\newcommand{\CommentVarTok}[1]{\textcolor[rgb]{0.56,0.35,0.01}{\textbf{\textit{#1}}}}
\newcommand{\ConstantTok}[1]{\textcolor[rgb]{0.00,0.00,0.00}{#1}}
\newcommand{\ControlFlowTok}[1]{\textcolor[rgb]{0.13,0.29,0.53}{\textbf{#1}}}
\newcommand{\DataTypeTok}[1]{\textcolor[rgb]{0.13,0.29,0.53}{#1}}
\newcommand{\DecValTok}[1]{\textcolor[rgb]{0.00,0.00,0.81}{#1}}
\newcommand{\DocumentationTok}[1]{\textcolor[rgb]{0.56,0.35,0.01}{\textbf{\textit{#1}}}}
\newcommand{\ErrorTok}[1]{\textcolor[rgb]{0.64,0.00,0.00}{\textbf{#1}}}
\newcommand{\ExtensionTok}[1]{#1}
\newcommand{\FloatTok}[1]{\textcolor[rgb]{0.00,0.00,0.81}{#1}}
\newcommand{\FunctionTok}[1]{\textcolor[rgb]{0.00,0.00,0.00}{#1}}
\newcommand{\ImportTok}[1]{#1}
\newcommand{\InformationTok}[1]{\textcolor[rgb]{0.56,0.35,0.01}{\textbf{\textit{#1}}}}
\newcommand{\KeywordTok}[1]{\textcolor[rgb]{0.13,0.29,0.53}{\textbf{#1}}}
\newcommand{\NormalTok}[1]{#1}
\newcommand{\OperatorTok}[1]{\textcolor[rgb]{0.81,0.36,0.00}{\textbf{#1}}}
\newcommand{\OtherTok}[1]{\textcolor[rgb]{0.56,0.35,0.01}{#1}}
\newcommand{\PreprocessorTok}[1]{\textcolor[rgb]{0.56,0.35,0.01}{\textit{#1}}}
\newcommand{\RegionMarkerTok}[1]{#1}
\newcommand{\SpecialCharTok}[1]{\textcolor[rgb]{0.00,0.00,0.00}{#1}}
\newcommand{\SpecialStringTok}[1]{\textcolor[rgb]{0.31,0.60,0.02}{#1}}
\newcommand{\StringTok}[1]{\textcolor[rgb]{0.31,0.60,0.02}{#1}}
\newcommand{\VariableTok}[1]{\textcolor[rgb]{0.00,0.00,0.00}{#1}}
\newcommand{\VerbatimStringTok}[1]{\textcolor[rgb]{0.31,0.60,0.02}{#1}}
\newcommand{\WarningTok}[1]{\textcolor[rgb]{0.56,0.35,0.01}{\textbf{\textit{#1}}}}
\usepackage{graphicx}
\makeatletter
\def\maxwidth{\ifdim\Gin@nat@width>\linewidth\linewidth\else\Gin@nat@width\fi}
\def\maxheight{\ifdim\Gin@nat@height>\textheight\textheight\else\Gin@nat@height\fi}
\makeatother
% Scale images if necessary, so that they will not overflow the page
% margins by default, and it is still possible to overwrite the defaults
% using explicit options in \includegraphics[width, height, ...]{}
\setkeys{Gin}{width=\maxwidth,height=\maxheight,keepaspectratio}
% Set default figure placement to htbp
\makeatletter
\def\fps@figure{htbp}
\makeatother
\setlength{\emergencystretch}{3em} % prevent overfull lines
\providecommand{\tightlist}{%
  \setlength{\itemsep}{0pt}\setlength{\parskip}{0pt}}
\setcounter{secnumdepth}{-\maxdimen} % remove section numbering
\ifLuaTeX
  \usepackage{selnolig}  % disable illegal ligatures
\fi
\IfFileExists{bookmark.sty}{\usepackage{bookmark}}{\usepackage{hyperref}}
\IfFileExists{xurl.sty}{\usepackage{xurl}}{} % add URL line breaks if available
\urlstyle{same} % disable monospaced font for URLs
\hypersetup{
  pdftitle={HW06 Spatial Statistics},
  pdfauthor={Doug Nychka},
  hidelinks,
  pdfcreator={LaTeX via pandoc}}

\title{HW06 Spatial Statistics}
\author{Doug Nychka}
\date{2022-11-12}

\begin{document}
\maketitle

\hypertarget{golden-hourly-ozone-2021}{%
\section{Golden hourly ozone 2021}\label{golden-hourly-ozone-2021}}

The goal of this activity is to build a functional linear model to
forecast tomorrows hourly ozone from today's record. Keep in mind that
although the basic model will be forecasting the basis coefficients our
final goal is a good forecast of the 24 hour cycle of ozone for the next
day. Forecasts of the coefficients is really just an intermediate step.

\begin{Shaded}
\begin{Highlighting}[]
\FunctionTok{suppressMessages}\NormalTok{(}\FunctionTok{library}\NormalTok{( fields))}
\end{Highlighting}
\end{Shaded}

\begin{verbatim}
## Warning: package 'fields' was built under R version 4.1.3
\end{verbatim}

\begin{verbatim}
## Warning: package 'spam' was built under R version 4.1.3
\end{verbatim}

\begin{Shaded}
\begin{Highlighting}[]
\FunctionTok{setwd}\NormalTok{(}\StringTok{"\textasciitilde{}/School/MATH498/Test2/"}\NormalTok{)}
\end{Highlighting}
\end{Shaded}

First some setup based a complete data record where missing hours have
been infilled using Tps.

\begin{Shaded}
\begin{Highlighting}[]
\NormalTok{tDay}\OtherTok{\textless{}{-}} \DecValTok{1}\SpecialCharTok{:}\DecValTok{365}
\FunctionTok{load}\NormalTok{(}\StringTok{"O3CompleteExample.rda"}\NormalTok{)}
\CommentTok{\# SVD of data,  day by hour}
\NormalTok{lookSVD}\OtherTok{\textless{}{-}} \FunctionTok{svd}\NormalTok{( O3Complete)}
\CommentTok{\#print( lookSVD$d)}
\NormalTok{D}\OtherTok{\textless{}{-}} \FunctionTok{diag}\NormalTok{(lookSVD}\SpecialCharTok{$}\NormalTok{d)}
\CommentTok{\# adjust signs to be more interpretable}
\CommentTok{\# }
\NormalTok{V}\OtherTok{\textless{}{-}} \SpecialCharTok{{-}}\DecValTok{1}\SpecialCharTok{*}\NormalTok{lookSVD}\SpecialCharTok{$}\NormalTok{v}
\NormalTok{U}\OtherTok{\textless{}{-}} \SpecialCharTok{{-}}\DecValTok{1}\SpecialCharTok{*}\NormalTok{lookSVD}\SpecialCharTok{$}\NormalTok{u}
\CommentTok{\# weighted basis functions (columns)}
\NormalTok{VBasis}\OtherTok{\textless{}{-}}\NormalTok{ V }\SpecialCharTok{\%*\%}\NormalTok{D}
\end{Highlighting}
\end{Shaded}

Just for fun and to get started here are the first four basis functions.

\begin{Shaded}
\begin{Highlighting}[]
\FunctionTok{fields.style}\NormalTok{()}
\FunctionTok{matplot}\NormalTok{( }\DecValTok{1}\SpecialCharTok{:}\DecValTok{24}\NormalTok{, VBasis[,}\DecValTok{1}\SpecialCharTok{:}\DecValTok{4}\NormalTok{], }\AttributeTok{type=}\StringTok{"l"}\NormalTok{, }\AttributeTok{lty=}\DecValTok{1}\NormalTok{, }\AttributeTok{lwd=}\DecValTok{2}\NormalTok{,}
         \AttributeTok{xlab=}\StringTok{"hour"}\NormalTok{, }\AttributeTok{ylab=}\StringTok{"basis function"}\NormalTok{)}
\FunctionTok{title}\NormalTok{(}\StringTok{"Basis functions}
\StringTok{      1 orange, 2 green, 3 blue, 4 red"}\NormalTok{)}
\end{Highlighting}
\end{Shaded}

\includegraphics{InClassForecastGoldenOzone_files/figure-latex/unnamed-chunk-3-1.pdf}

Here are the responses and lagged quantities to use both in terms of the
hourly data and also in terms of the coefficients. Note that we are
trying forecast the full day's cycle: all 24 values of O3 for the next
day.

\begin{Shaded}
\begin{Highlighting}[]
\NormalTok{N}\OtherTok{\textless{}{-}} \FunctionTok{nrow}\NormalTok{(O3Complete )}
\NormalTok{tm}\OtherTok{\textless{}{-}}\NormalTok{ tDay[}\SpecialCharTok{{-}}\NormalTok{N] }\CommentTok{\# can only forecast up to the second to last day!}
\NormalTok{Y}\OtherTok{\textless{}{-}}\NormalTok{ O3Complete[}\SpecialCharTok{{-}}\DecValTok{1}\NormalTok{,] }\CommentTok{\# Tommorrows O3}
\NormalTok{X}\OtherTok{\textless{}{-}}\NormalTok{  O3Complete[}\SpecialCharTok{{-}}\NormalTok{N,] }\CommentTok{\# Todays O3}
\NormalTok{XU}\OtherTok{\textless{}{-}}\NormalTok{ U[}\SpecialCharTok{{-}}\NormalTok{N,] }\CommentTok{\# todays coefficients}
\NormalTok{YU}\OtherTok{\textless{}{-}}\NormalTok{ U[}\SpecialCharTok{{-}}\DecValTok{1}\NormalTok{,] }\CommentTok{\# tomorows coefficents}
\end{Highlighting}
\end{Shaded}

\hypertarget{variance-explained.}{%
\section{Variance explained.}\label{variance-explained.}}

Based on the singular values we have the variance explained by each
basis function and also the percent variance explained. We see the
variance tends to decrease after 3 to 4 singular values.

\begin{Shaded}
\begin{Highlighting}[]
\FunctionTok{print}\NormalTok{(  }\FunctionTok{round}\NormalTok{(lookSVD}\SpecialCharTok{$}\NormalTok{d}\SpecialCharTok{\^{}}\DecValTok{2}\NormalTok{,}\DecValTok{0}\NormalTok{) )}
\end{Highlighting}
\end{Shaded}

\begin{verbatim}
##  [1] 18303428   183750    77652    36155    30064    21093    12080     8178
##  [9]     6022     5262     3986     3103     2837     2194     1711     1647
## [17]     1317     1192      988      829      786      670      626      509
\end{verbatim}

\begin{Shaded}
\begin{Highlighting}[]
\CommentTok{\# and normalized as }
\FunctionTok{print}\NormalTok{(  }\DecValTok{100}\SpecialCharTok{*}\FunctionTok{cumsum}\NormalTok{(lookSVD}\SpecialCharTok{$}\NormalTok{d}\SpecialCharTok{\^{}}\DecValTok{2}\NormalTok{)}\SpecialCharTok{/} \FunctionTok{sum}\NormalTok{( lookSVD}\SpecialCharTok{$}\NormalTok{d}\SpecialCharTok{\^{}}\DecValTok{2}\NormalTok{) )}
\end{Highlighting}
\end{Shaded}

\begin{verbatim}
##  [1]  97.84749  98.82979  99.24491  99.43819  99.59890  99.71166  99.77624
##  [8]  99.81996  99.85215  99.88028  99.90159  99.91818  99.93335  99.94507
## [15]  99.95422  99.96303  99.97007  99.97644  99.98172  99.98615  99.99035
## [22]  99.99393  99.99728 100.00000
\end{verbatim}

This justifies using a fewer number of basis functions than 24!

\hypertarget{how-not-to-proceed}{%
\section{\texorpdfstring{How \emph{not} to
proceed}{How not to proceed}}\label{how-not-to-proceed}}

If we just predicted every hour for the next day using the 24 ozone
values from the previous day how many total parameters would be in our
forecast model?

\hypertarget{getting-started}{%
\section{Getting started}\label{getting-started}}

How well do 4 basis functions approximate the full set of hourly values?
Recall that the approximation is

\begin{Shaded}
\begin{Highlighting}[]
\NormalTok{ApproxO3}\OtherTok{\textless{}{-}}\NormalTok{ YU[,}\DecValTok{1}\SpecialCharTok{:}\DecValTok{4}\NormalTok{]}\SpecialCharTok{\%*\%} \FunctionTok{t}\NormalTok{(VBasis[,}\DecValTok{1}\SpecialCharTok{:}\DecValTok{4}\NormalTok{])}
\end{Highlighting}
\end{Shaded}

and compare to \texttt{Y} and report the RMSE for the difference. We
will call this the \emph{oracle} forecast because it assumes we can
predict tomorrows coefficients perfectly.

\hypertarget{forecast-of-first-coefficient}{%
\section{forecast of first
coefficient}\label{forecast-of-first-coefficient}}

Build an lm model to predict tommorows first ``U'' coefficient,
\texttt{YU{[},1{]}}, based on today's 4 ``U'' coefficients
\texttt{XU{[},1:4{]}}.

Evaluate the forecast RMSE by expanding with the first basis function
and compare to \texttt{Y}. Compare this RMSE to how well the
\emph{oracle} forecast RMSE.

\hypertarget{forecast-of-all-4-coefficients}{%
\section{forecast of all 4
coefficients}\label{forecast-of-all-4-coefficients}}

Now repeat using 4 lm models to forecast all 4 coefficients. Again how
well does it do?

\hypertarget{model-checking}{%
\section{model checking}\label{model-checking}}

Find the RMSE for your forecast for each day and plot these over time.
Comment on any patterns.

\emph{Hint} use the apply function or \textbf{rowSums} to find the MSE
for each day separately.

\end{document}
